\documentclass{article}
\usepackage[utf8]{inputenc}
\usepackage{authoraftertitle}

%% custom style file
\usepackage{notes}

%%  math symbols
\usepackage{amsmath}    % ams math
\usepackage{amssymb}    % ^^ 
\usepackage{amsthm}     % ^^
\usepackage{mathtools}  % additional tools
\usepackage{centernot}  % nice /not commands
\usepackage{xfrac}      % nice fractions
\usepackage{bbm}        % indicators

\usepackage{enumitem}   % better enums
\usepackage{hyperref}   % internal referencing

\usepackage{parskip}    % paragraph formatting
\usepackage{todonotes}  % inline todo notes 

\addbibresource{refs.bib}

%% load macros and pre-defined symbols
\input{macros/math}
%% macros for paragraph mode
\input{macros/paragraph}

%% (optional) load tikz and PGFplots
% \input{macros/plots}

\title{Approximating ReLU Networks by Conic Decomposition}
\author{Aaron Mishkin\\ \texttt{amishkin@cs.stanford.edu}}
\date{}

\begin{document}

\maketitle

\section{Introduction}

Recall the convex re-formulation of a two-layer ReLU MLP is
\begin{equation}\label{eq:convex-relu-mlp}
	\begin{aligned}
		p^* = \min_{v, w} \, & \mathcal{L}\Big(  \sum_{D_i \in \tilde \calD}^P   D_i X (v_i  -  w_i), y\Big) + \lambda \sum_{D_i \in \tilde \calD} \norm{v_i}_2 + \norm{w_i}_2 \\
		                     & \text{s.t.} \, \rbr{2 D_i - I_n} X v_i \succeq 0, \, \rbr{2 D_i - I_n} X w_i \succeq 0.
	\end{aligned}
\end{equation}
Introduce the parameter \( u_i = v_i - w_i \) to obtain \( u_i + w_i = v_i \) and the equivalent problem
\begin{equation}\label{eq:reparameterized-problem}
	\begin{aligned}
		p^* = \min_{u, w} \, & \mathcal{L}\Big(  \sum_{D_i \in \tilde \calD}^P   D_i X u_i, y\Big) + \lambda \sum_{D_i \in \tilde \calD} \norm{u_i + w_i}_2 + \norm{w_i}_2 \\
		                     & \text{s.t.} \, \rbr{2 D_i - I_n} X (u_i + w_i) \succeq 0, \, \rbr{2 D_i - I_n} X w_i \succeq 0.
	\end{aligned}
\end{equation}
The regularization terms for this problem are not easily handled, so we form the following relaxation:
\begin{equation}\label{eq:relaxation}
	\begin{aligned}
		r^* = \min_{u, w} \, & \mathcal{L}\Big(  \sum_{D_i \in \tilde \calD}^P   D_i X u_i, y\Big) + \lambda \sum_{D_i \in \tilde \calD} \norm{u_i}_2 \\
	\end{aligned}
\end{equation}
Triangle inequality implies \( \norm{u_i}_2 = \norm{u_i + w_i - w_i}_2 \leq \norm{u_i + w_i}_2 + \norm{w_i}_2 \), from which we deduce \( r^* \leq p^* \).
Finally, we observe that Problem~\ref{eq:relaxation} is equivalent to the Gated ReLU training problem when the constraints are dropped.

\section{Conic Decompositions}

For each \( D_i \in \calD_X \), let \( \calK_i = \cbr{w : (2D_i - I) X w \geq 0} \).
Since there exists \( w_i \) generating \( D_i \),
\[
	D_i = \text{diag}\rbr{\mathbbm{1}\rbr{X w_i > 0}},
\]
the cone \( \calK_i \) is non-empty.
Moreover, it is clearly closed and convex.
Thus, \( \text{relint} (\calK_i) \) is non-empty.
By definition, \( u_i = v_i - w_i \), where \( w_i, v_i \in \calK_i \).
That is, \( u_i \in \calK_i - \calK_i \).
It is natural to ask if \( \calK_i - \calK_i = \R^d \), in which case \( u_i \) is unconstrained.

\begin{lemma}\label{lemma:affine-characterization}
	The affine dimension of \( \calK_i \) is \( d \) if and only if \( \calK_i - \calK_i = \R^d \).
\end{lemma}
\begin{proof}
	Let \( x \in \text{aff} (\calK_i) \).
	Then \( x = \sum_{j=1}^m \alpha_j y_j \), \( y_j \in \calK_i \).
	If \( \alpha_j \geq 0 \), then \( \alpha_j y_j \in \calK_i \) since \( \calK_i \) is a cone.
	Otherwise, \( \alpha_j y_j \in - \calK_i \).
	An inductive argument implies \( x \in \calK_i - \calK_i \) and \( \text{aff} (\calK_i) \subseteq \calK_i - \calK_i \).
	The reverse inclusion is trivial, so we conclude that \( \text{aff} (\calK_i) = \calK_i - \calK_i \).
	Since \( 0 \in \text{aff} (\calK_i) \), this set is actually a subspace.
	If \( \text{dim} \; \text{aff} (\calK_i) = d \), then we must have \( \text{aff} (\calK_i) = \calK_i - \calK_i = \R^d \).

	If \( \text{dim} \; \text{aff} (\calK_i) < d \), then \( \text{aff} (\calK_i) = \calK_i - \calK_i \subset \R^d \) must hold and we have shown the reverse implication by the contrapositive.
\end{proof}

Unfortunately, it is possible that the affine hull of \( K_i \) is only a subspace.
Let \( d = 2 \) and consider the vectors \( x_1 = [1, 0], x_2 = [-1, 0], x_3 = [0, 1] \).
The activation pattern \( D_i = \text{diag}\rbr{0, 0, 1} \) is attained only in the subspace \( \cbr{ w : w_1 = 0 } \).
In particular,
\[
	\calK_i = \cbr{w : w_1 = 0, w_2 \geq 0}.
\]
which does not have affine dimension \( d \).
\autoref{lemma:affine-characterization} completes the counter-example.

We can establish \( \calK_i - \calK_i = \R^d \) under the restrictive condition that \( n \geq d \) and \( X \) is full-rank.
That is, the rows of \( X \) are linearly independent.
\begin{proposition}\label{prop:linearly-independent-cone}
	Suppose that \( n \geq d \) and \( X \) is full-rank.
	Then \( \calK_i - \calK_i = \R^d \).
\end{proposition}
\begin{proof}
	Let \( \bar w \in \text{relint} (\calK_i) \), which exists since the relative interior is non-empty.
	Assume that \( [D_i]_{jj} x_j^\top \bar w = 0 \) for at least one \( j \in [n] \) and let \( \tilde X \) denote the sub-matrix formed by the rows of \( X \) for which this inequality is tight.
	Since \( X \) is full-rank and \( n \geq d \), the rows of \( X \) are linearly independent and the rows of \( \tilde X \) are trivially linearly independent.
	Let \( x_k \) be an arbitrary row of \( \tilde X \) (noting \( x_k \neq 0 \) by linear independent) and define \( z_k \) to be the component of \( x_k \) which is orthogonal to the remaining rows of \( \tilde X \).
	Clearly such a vector exists since the rows of \( \tilde X \) are linearly independent.
	Now, define \( w' = \bar w + [D_i]_{jj} z_k \) to obtain
	\[
		[D_i]_{kk} x_k^\top w' = [D_i]_{kk} x_k^\top \bar w + \norm{z_k}_2^2 = \norm{z_k}_2^2 > 0,
	\]
	and, for \( j \neq k \),
	\[
		[D_i]_{jj} x_j^\top w' = [D_i]_{jj} x_{j}^\top \bar w +  [D_i]_{jj} [D_i]_{kk} x_{j}^\top z_k = 0.
	\]
	This contradicts \( \bar w \in \text{relint} (\calK_i) \).
	We deduce that \( [D_i]_{jj} x_j^\top \bar w > 0 \) for all \( j \in [n] \).

	Noting that the rows of \( X \) span an \( n \)-dimensional subspace by linear independence,
	let \( y_{1}, \ldots, y_{d-n} \) be arbitrary vectors spanning the orthogonal complement, \( \text{row}(X)^\perp \).
	Clearly \( \bar w + y_{k} \in \calK_i \) for each \( k \in [d-n] \) since \( x_j \perp y_k \) for all \( j, k \in [n] \times [d-n] \).
	Moreover, for \( \epsilon > 0 \) small enough, \( \bar w + \epsilon x_{j} \in \calK_i \) for each \( j \in [n] \) since \( \bar w \) is a relative interior point for which each of the cone constraints holds strictly.
	Thus, \( \calK_i - \bar w\) contains a basis for \( \R^d \) and \( \text{dim aff} (\calK_i) = d \).
	\autoref{lemma:affine-characterization} now gives the desired result.

\end{proof}

\autoref{prop:linearly-independent-cone} requires \( X \) to have full row-rank, which is fairly restrictive. 
However, avoiding the counterexample we gave earlier requires only that no two rows of \( X \) are negatively collinear.
The main difficulty in extending \autoref{prop:linearly-independent-cone} to this more general setting is showing that no constraints are tight at \( \bar w \).

\printbibliography[]

\end{document}
