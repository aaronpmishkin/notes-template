\documentclass{article}
\usepackage[utf8]{inputenc}
\usepackage{authoraftertitle}

%% custom style file
\usepackage{notes}

%%  math symbols
\usepackage{amsmath}    % ams math
\usepackage{amssymb}    % ^^ 
\usepackage{amsthm}     % ^^
\usepackage{mathtools}  % additional tools
\usepackage{centernot}  % nice /not commands
\usepackage{xfrac}      % nice fractions
\usepackage{bbm}        % indicators

\usepackage{enumitem}   % better enums
\usepackage{hyperref}   % internal referencing
\usepackage[noabbrev, capitalize]{cleveref}   % internal referencing

\usepackage{parskip}    % paragraph formatting
\usepackage{todonotes}  % inline todo notes 

\addbibresource{refs.bib}

%% load macros and pre-defined symbols
\input{macros/math}
%% macros for paragraph mode
\input{macros/paragraph}


\newcommand{\Wone}{W^{(1)}}
\newcommand{\Wl}{W^{(l)}}

\newcommand{\wone}{W^{(1)}}
\newcommand{\wtwo}{W^{(2)}}
\newcommand{\wthree}{W^{(3)}}

\newcommand{\Dl}{D^{(l)}}
\newcommand{\Dzero}{D^{(0)}}
\newcommand{\Done}{D^{(1)}}
\newcommand{\Dtwo}{D^{(2)}}

\newcommand{\calDl}{\calD^{(l)}}
\newcommand{\calDzero}{\calD^{(0)}}
\newcommand{\calDone}{\calD^{(1)}}
\newcommand{\calDtwo}{\calD^{(2)}}

\newcommand{\calKl}{\calK^{(l)}}
\newcommand{\calKzero}{\calK^{(0)}}

\newcommand{\Xl}{X^{(l)}}
\newcommand{\Xzero}{X^{(0)}}
\newcommand{\Xone}{X^{(1)}}

\newcommand{\Zl}{Z^{(l)}}
\newcommand{\Zzero}{Z^{(0)}}
\newcommand{\Zone}{Z^{(1)}}

%% (optional) load tikz and PGFplots
% \input{macros/plots}

\title{Convex Reformulations of Deep Neural Networks}
\author{Aaron Mishkin\\ \texttt{amishkin@cs.stanford.edu}}
\date{}

\begin{document}

\maketitle

\section{Introduction}

Let \( X \in \R^{n \times d_0} \) be a matrix of observations and \( y \in
\R^{n \times c} \) a matrix of \( c \)-dimensional responses for each
observation.
This setup covers both multi-class classification and regression with
vector-valued targets.
We are interested fully-connected ReLU neural networks with arbitrarily many
layers.
Let \( \Wl \in \R^{d_{l-1} \times d_l} \) be a matrix of weights.
The activations at layer $l$ are defined to be,
\begin{equation}
    \Zl = \rbr{Z^{(l-1)} \Wl}_+,
\end{equation}
where we take \( \Zzero = X \) and \(
\rbr{\cdot}_+ \) is ReLU activation function defined element-wise as \(
\rbr{z}_+ = \max\cbr{z, 0} \).
Then a fully-connected neural network with \( k \)-layers has the prediction
function given by,
\begin{equation}
    f_\Theta(X) = Z_{(k-1)}
    W^{(k)},
\end{equation}
where \( W^{(k)} \in \R^{d_{k-1} \times c} \) ensures
that the prediction is of the right dimension.
Here, \( \Theta = \cbr{\Wone, \ldots, W^{(k)}} \) denotes the collection of
weights for the network.
If \( k = 3 \), this recursive definition can be unrolled to give an
explicit predictor function,
\begin{equation}
    f_\Theta(X) = \rbr{\rbr{X W^{(1)}}_+ W^{(2)}}_+ W^{(3)}.
\end{equation}
In this note, we focus on models trained using the squared-error loss function
and weight-decay regularization.
That is, problems of the form,
\begin{equation}
    \Theta^* \in \argmin_{\Theta} \half \norm{f_\Theta(X) - y}_2^2 +
    \frac{\lambda}{2} \sum_{l=1}^k \norm{\Wl}_F^2,
\end{equation}
where \( \lambda \geq 0 \) controls the strength of the regularization.

Our analysis will work by considering the space of activation patterns that
can be attained by a single ReLU neuron.
Given a matrix \( A \in \R^{a, b} \), set of activation patterns for \( A \) is
\begin{equation}
    \calD(A) = \cbr{ \text{diag}\rbr{\mathbbm{1}\rbr{ X w \geq 0}}: w \in \R^d }.
\end{equation}
Note that each pattern \( D_i \in \calD(A) \) is a diagonal matrix with \(
[D_i]_{jj} = 1 \) if \( x_i^\top w \geq 0 \) and \( 0 \) otherwise.
Each activation pattern is associated with a convex cone of vectors which
are consistent with that pattern,
\begin{equation}
    \calK_i(A) = \cbr{w : (2D_i - I) X w \geq 0 }.
\end{equation}
It is straightforward to show that \( w \in \calK_i(A) \) if and only if \(
(Aw)_+ = D_i A w \), i.e. \( D_i \) linearizes the ReLU activation.

To reduce the burden of notation, let \( \Xzero = X \) and \( \calDzero =
\calD(X) \).
Let \( \Dzero_i \in \calDzero \) denote the activation patterns associated
with \( \Xzero \), \( \calKzero_i \) the associated cone.
The total number of activation patterns is \( p_0 = \abs{\calDzero} \).
Let \( a_0 = d_0 \) so that \( \Xzero \in \R^{n \times a_0} \).
Then, we can recursively define,
\begin{equation}
    \Xl =
    \begin{bmatrix}
        D_0^{(l-1)} X^{(l-1)} & \ldots & D^{(l-1)}_{p_{(l-1)}} X^{(l-1)}
    \end{bmatrix}
    ,
\end{equation}
\( \calDl = \calD(\Xl) \), \( \Dl_i \in \calDl \) where \(
\calKl_i \) is the associated cone, and \( p_l = \abs{\calD^l} \).
Note that \( \Xl \in \R^{n \times p_{l-1} \cdot a_{l-1}} \), which implies \(
a_l = p_{l-1} \cdot a_{l-1} \).
These objects will allow us to reason about activation patterns in the \(
l^{\text{th}} \) layer of a \( k \)-layer network.

\section{Layer Elimination Lemma}

Our goal in this section is to develop a basic lemma which merges layers in
the neural network.
Iteratively applying this lemma will then allow us to derive a convex
reformulation.

\begin{lemma}
    Let \( v^{(l)} \in \R^{d_l \cdot a_{l-1}} \) such that
    \( v^{(l)} \in \calC^{(l)} \), where \( \calC^{(l)} \) is a closed convex
    cone.
    Assume \( d_l \geq d_{l+1} a_l p_l \).
    Then the activations at layer \( l+1 \) are given by,
    \begin{equation}\label{eq:assumed-structure}
        Z_{j}^{(l+1)} = \rbr{\sum_{i = 1}^{d_{l}} (X^{(l-1)} v^{(l)}_i)_+
        W_{ij}^{(l+1)}}_+.
    \end{equation}
    if and only if the post-activations are also equal to,
    \begin{equation}
        Z_{j}^{(l+1)} = \rbr{X^{(l)} v^{(l+1)}_{j}}_+,
    \end{equation}
    for some
    \begin{equation}
        \begin{aligned}
            v^{(l+1)} \in \calC^{(l+1)}
            :=
            \text{Cone} \bigg\{
            v^{(l+1)} \in \R^{d_{l+1} \cdot a_l} :
             & \exists \, v^{(l)} \in \calC^{(l)}
            \text{ satisfying } v_{i}^{(l)} \in \calKl_i, \,                 \\
             & \exists \, W^{(l+1)} \in \R^{d_l \times d_{l+1}},             \\
             & \text{ s.t. } v^{(l+1)}_{ij} = v^{(l)}_{i} \cdot W_{ij}^{(l)}
            \bigg\}.
        \end{aligned}
    \end{equation}
\end{lemma}
\begin{proof}
    We start with the forward direction.
    Let
    \begin{equation}
        v^{(l+1)}_{j} =
        \begin{bmatrix}
            \sum_{m \in \calI^{(l)}_0} v^{(l)}_{m} \cdot W_{mj}^{(l+1)} \\
            \vdots                                                      \\
            \sum_{m \in \calI^{(l)}_{p_l}} v^{(l)}_{m} \cdot W_{m j}^{(l+1)}
        \end{bmatrix}
        \quad \text{s.t.} \quad
        \calI^{(l)}_{i} = \cbr{j \in [d_l] :
            \Dl_i = \text{diag}\rbr{\mathbbm{1}(X^{(l-1)} v^{(l)}_j
                \geq 0)}}.
    \end{equation}
    By re-writing \( v^{(l+1)} \) as a sum over blocks defined by \( \calDl_i \),
    \[
        v^{(l+1)}_{j} = \sum_{i=1}^{p_l}
        \sum_{m \in \calI_i^{(l)}}
        \begin{bmatrix}
            0                        \\
            \vdots                   \\
            v_m^{(l)} W_{mj}^{(l+1)} \\
            \vdots                   \\
            0
        \end{bmatrix},
    \]
    we see that \( v^{(l+1)} \) is a conic combination of vectors in \( \calC^{(l+1)} \).
    Since \( \calC^{(l+1)} \) is itself a convex cone, we conclude
    \( v^{(l+1)} \in \calC^{(l+1)} \) as required.

    For convenience in the following computation, define
    \( v_{ij}^{(l)} = \sum_{m \in \calI^{(l)}_i} v^{(l)}_{m} \cdot W_{mj}^{(l+1)} \).
    By assumption, the activations are given by
    \begin{align*}
        Z_{j}^{(l+1)}
         & = \rbr{\sum_{i = 1}^{d_{l}} (X^{(l-1)} v^{(l)}_i)_+
        W_{ij}^{(l+1)}}_+                                                \\
        \intertext{
            For each \( v^{(l)}_i \), there exists an
            activation pattern
            \( \Dl_i = \text{diag}\rbr{\mathbbm{1}(X^{(l-1)} v^{(l)}_i \geq 0)} \)
            such that \( \Dl_i X^{(l-1)} v^{(l)}_i = \rbr{X^{(l-1)} v^{(l)}_i}_+ \).
            Thus, we obtain
        }
        Z_{j}^{(l+1)}
         & = \rbr{\sum_{i = 1}^{d_{l}} \Dl_i X^{(l-1)}
        v^{(l)}_i W_{ij}^{(l+1)}}_+                                      \\
         & = \rbr{\sum_{i = 1}^{p_{l}} \Dl_i X^{(l-1)}
        \sum_{m \in \calI^{(l)}_{i}} \sbr{v^{(l)}_m W_{mj}^{(l+1)}}}_+   \\
         & = \rbr{\sum_{i = 1}^{p_{l}} \Dl_i X^{(l-1)} v^{(l+1)}_{ij}}_+ \\
         & = \rbr{X^{(l)} v^{(l+1)}_{j}}_+.
    \end{align*}
    That completes the forward direction.

    Now we prove the reverse implication.
    Let \( v^{(l+1)} \in \calC^{(l+1)} \).
    Then \( v^{(l+1)} \in \R^{d_{l+1} \cdot a_l} \).
    By Carathéodory's theorem, every vector in \( \calC^{(l+1)} \) can be
    represented as the conic combination of \( d_{l+1} \cdot a_l \) vectors
    in
    \[
        \begin{aligned}
            \calE^{(l+1)}
            =
            \bigg\{
            v^{(l+1)} \in \R^{d_{l+1} \cdot a_l} :
             & \exists \, v^{(l)} \in \calC^{(l)}
            \text{ satisfying } v_{i}^{(l)} \in \calKl_i, \,                 \\
             & \exists \, W^{(l+1)} \in \R^{d_l \times d_{l+1}},             \\
             & \text{ s.t. } v^{(l+1)}_{ij} = v^{(l)}_{i} \cdot W_{ij}^{(l)}
            \bigg\}.
        \end{aligned}
    \]
    Thus, we obtain the representation,
    \begin{equation}
        v^{(l+1)}
        = \sum_{m=1}^{d_{l+1} a_l} u_m^{(l+1)},
        \quad \quad u_m^{(l+1)} \in \calE^{(l+1)}
        \text{ for all } m \in \cbr{1, \ldots, d_{l+1} a_l}.
    \end{equation}
    Using this decomposition in the assumed expression for the activations,
    \begin{align*}
        Z_{j}^{(l+1)}
         & = \rbr{X^{(l)} v^{(l+1)}_{j}}_+                              \\
         & = \rbr{\sum_{m=1}^{d_{l+1} a_l} X^{(l)} u^{(l+1)}_{m, j}}_+  \\
         & = \rbr{\sum_{m=1}^{d_{l+1} a_l} \sum_{i=1}^{p_l}
        D_{i}^{(l)} X^{(l-1)} u^{(l+1)}_{m, i, j}}_+                    \\
         & = \rbr{\sum_{m=1}^{d_{l+1} a_l} \sum_{i=1}^{p_l}
        D_{i}^{(l)} X^{(l-1)} u^{(l)}_{m, i} \cdot W_{m, ij}^{(l+1)}}_+ \\
         & = \rbr{\sum_{m=1}^{d_{l+1} a_l} \sum_{i=1}^{p_l}
        \rbr{X^{(l-1)} u^{(l)}_{m, i}}_+ \cdot W_{m, ij}^{(l+1)}}_+,
    \end{align*}
    which matches \cref{eq:assumed-structure} if \( d_l \geq d_{l+1} a_l p_l \).

    \begin{itemize}
        \item
              \textcolor{red}{Is there a representer theorem that we can use to
                  bound the number of neurons?}

        \item
              \textcolor{red}{This required number of neurons is very loose in
                  the two-layer scalar setting.  Can we use a version of
                  Caratheodory's theorem restricted to an affine subspace to improve
                  this?}
    \end{itemize}

\end{proof}

\newpage

\section{Old Notes}

Consider the three-layer neural network with ReLU activation functions,
\begin{equation}
    \label{eq:three-layer-mlp}
    f(X)
    = ((X W^{(1)})_+ W^{(2)})_+ w^{(3)}
    = \sum_{j = 1}^{m_2} \rbr{\sum_{i = 1}^{m_1} \rbr{X \wone_i}_+ \wtwo_{ji}}_+ \wthree_j,
\end{equation}
where \( \wone_i \in \R^{d} \), \( \wtwo_j \in \R^{m_1} \) and \( \wthree \in \R^{m_2} \).
Given training set \( X, y \), the optimization problem for this neural network
with least-squares loss and squared \( \ell_2 \) regularization is given by
\begin{equation}
    \label{eq:three-layer-problem}
    \min
    \half \left\|
    \sum_{j = 1}^{m_2} \rbr{\sum_{i = 1}^{m_1} \rbr{X \wone_i}_+ \wtwo_{ji}}_+ \wthree_j
    - y
    \right\|_2^2
    + \frac{\lambda}{2}
    \sum_{i=1}^{m_1} \norm{\wone_i}_2^2
    + \sum_{j=1}^{m_2} \norm{\wtwo_j}_2^2
    + \sum_{j=1}^{m_2} \norm{\wthree_j}_2^2.
\end{equation}
Let's look at the regularization terms.
First, observe that by changing the index of summation, \[ \sum_{j=1}^{m_2}
    \norm{\wtwo_j}_2^2 = \sum_{i=1}^{m_1} \norm{\wtwo_{\cdot, i}}_2^2.
\]
Now applying Young's inequality twice, we obtain
\begin{align*}
    2 \sum_{j = 1}^{m_2} \norm{\wtwo_j}_2 \norm{\wthree_j}_2
     & \leq \sum_{j=1}^{m_2} \norm{\wtwo_j}_2^2
    + \sum_{j=1}^{m_2} \norm{\wthree_j}_2^2     \\
    2 \sum_{i = 1}^{m_1} \norm{\wone_i}_2 \norm{\wtwo_{\cdot, i}}_2
     & \leq \sum_{i=1}^{m_1} \norm{\wone_i}_2^2
    + \sum_{i=1}^{m_1} \norm{\wtwo_{\cdot, i}}_2^2.
\end{align*}
Equality can be achieved by choosing \( \norm{\wtwo_j}_2 = \norm{\wthree_j}_2
\) for every \( j \in [m_2] \) and \( \norm{\wone_i}_2 = \norm{\wtwo_{\cdot,
        i}}_2 \) for every \( i \in [m_1] \), respectively.
Since the ReLU activation function is positively homogeneous, we deduce that
any optimal solution \( \rbr{\wone, \wtwo, \wthree} \) must satisfy these
equalities as otherwise the regularization term could be decreased without
changing the loss.

Using this observation, we rewrite the training problem as the following
constrained program:
\begin{align*}
    \min
     & \half \left\|
    \sum_{j = 1}^{m_2} \rbr{\sum_{i = 1}^{m_1} \rbr{X \wone_i}_+ \wtwo_{ji}}_+ \wthree_j
    - y
    \right\|_2^2
    + \frac{\lambda}{2}
    \sum_{i=1}^{m_1} \norm{\wone_i}_2^2
    + \sum_{j=1}^{m_2} \norm{\wtwo_j}_2^2
    + \sum_{j=1}^{m_2} \norm{\wthree_j}_2^2                     \\
     & =
    \min_{\norm{\wtwo_j}_2 = \norm{\wthree_j}_2}
    \half \left\|
    \sum_{j = 1}^{m_2} \rbr{\sum_{i = 1}^{m_1} \rbr{X \wone_i}_+ \wtwo_{ji}}_+ \wthree_j
    - y
    \right\|_2^2
    + \frac{\lambda}{2}
    \sum_{i=1}^{m_1} \norm{\wone_i}_2^2
    + 2\sum_{j=1}^{m_2} \norm{\wtwo_j}_2^2                      \\
     & =
    \min_{\norm{\wtwo_j}_2 = \norm{\wthree_j}_2}
    \half \left\|
    \sum_{j = 1}^{m_2} \rbr{\sum_{i = 1}^{m_1} \rbr{X \wone_i}_+ \wtwo_{ji}}_+ \wthree_j
    - y
    \right\|_2^2
    + \frac{\lambda}{2}
    \sum_{i=1}^{m_1} \norm{\wone_i}_2^2
    + 2\sum_{i=1}^{m_1} \norm{\wtwo_{\cdot, i}}_2^2             \\
     & =
    \min_{\norm{\wtwo_j}_2 = \norm{\wthree_j}_2}
    \half \left\|
    \sum_{j = 1}^{m_2} \rbr{\sum_{i = 1}^{m_1} \rbr{X \wone_i}_+ \wtwo_{ji}}_+ \wthree_j
    - y
    \right\|_2^2
    + \sqrt{2}\lambda
    \sum_{i=1}^{m_1} \norm{\wone_i}_2 \norm{\wtwo_{\cdot, i}}_2 \\
     & =
    \min_{\norm{\wtwo_j}_2 = \norm{\wthree_j}_2 = 1}
    \half \left\|
    \sum_{j = 1}^{m_2} \rbr{\sum_{i = 1}^{m_1} \rbr{X \wone_i}_+ \wtwo_{ji}}_+ \wthree_j
    - y
    \right\|_2^2
    + \sqrt{2}\lambda
    \sum_{i=1}^{m_1} \norm{\wone_i}_2,
\end{align*}
where the last equality again uses positive homogeneity of the ReLU activation
function.

Now let's look at linearizing the activation functions.
Let \( \Done_i = \text{diag}\rbr{\mathbbm{1}(X \wone_i \geq 0)} \).
Then, we have
\begin{align*}
    \rbr{\sum_{i = 1}^{m_1} \rbr{X \wone_i}_+ \wtwo_{ji}}_+
     & = \rbr{\sum_{i = 1}^{m_1} \Done_i X \wone_i \wtwo_{ji}}_+ \\
     & = \rbr{\sum_{i = 1}^{m_1} \Done_i X u_{ij}}_+,
\end{align*}
where \( u_{ij} = \wone_i \wtwo_{ji} \in \R^{d} \).
Define the feature matrix with first-order splits to be,
\[
    \Xone =
    \begin{bmatrix}
        \Done_1 X & \Done_2 X & \ldots \Done_{m_1}
        X
    \end{bmatrix}
    , \] and let \( u_j \in \R^{|d \times m_1|} \) be the
concatenation of the \( u_{ij} \) vectors.
Then the activation becomes
\begin{align*}
    \rbr{\sum_{i = 1}^{m_1} \Done_i X u_{ij}}_+
     & = \rbr{\Xone u_j}_+  \\
     & = \Dtwo_j \Xone u_j,
\end{align*}
where \( \Dtwo_j = \text{diag}\rbr{\mathbbm{1}(\Xone u_j \geq 0)} \).
Plugging this into the expression for the neural network prediction
function gives,
\begin{align*}
    f(X)
     & = \sum_{j = 1}^{m_2} \Dtwo_j \Xone u_j \wthree_j \\
     & = \sum_{j = 1}^{m_2} \Dtwo_j \Xone v_j,
\end{align*}
where \( v_j = u_j \cdot \wthree_j \).
We can plug this back into the training problem to obtain a new optimization
program,
\begin{equation}
    \begin{aligned}
        \min
         & \half \left\|
        \sum_{i = 1}^{m_1} \sum_{j = 1}^{m_2} \Dtwo_j \Done_i X v_{ij}
        - y
        \right\|_2^2
        + \sqrt{2}\lambda
        \sum_{i=1}^{m_1} \sum_{j = 1}^{m_2}
        \norm{v_{ij}}_2 |\wtwo_{ji}|      \\
         & \hspace{2em} \text{s.t.} \quad
        v_{ij} = \pm \wone_{i} \cdot \wtwo_{ji}, \quad \norm{\wtwo_{\cdot, i}}_2 = 1,
    \end{aligned}
\end{equation}
where we have used the fact that \( \sum_{j=1}^{m_2} \sbr{\wtwo_{ji}}^2 = 1 \) and
\[
    \sum_{i=1}^{m_1} \norm{\wone_i}_2
    = \sum_{i=1}^{m_1} \sum_{j=1}^{m_2} \norm{\wone_i}_2 \sbr{\wtwo_{ji}}^2
    = \sum_{i=1}^{m_1} \sum_{j=1}^{m_2} \norm{v_{ij}}_2 \wtwo_{ji}.
\]
Alternatively, we can define the regularizer,
\[
    R(v) =
    \begin{cases}
        \sum_{i=1}^{m_1}
        \norm{\wone_{i}}_2
         & \mbox{if \( \exists \wone, \wtwo,
        \norm{\wtwo_{\cdot, i}}_2 = 1 \) such that \( v_{ij} = \pm \wone_{i} \wtwo_{ji} \)} \\
        +\infty
         & \mbox{otherwise.}
    \end{cases}
\]
In this case, we can write the optimization problem slightly differently as
\begin{equation}
    \begin{aligned}
        \min
         & \half \left\|
        \sum_{i = 1}^{m_1} \sum_{j = 1}^{m_2} \Dtwo_j \Done_i X v_{ij}
        - y
        \right\|_2^2
        + \sqrt{2}\lambda R(v).
    \end{aligned}
\end{equation}

\section{Questions}:

\begin{itemize}
    \item It looks to me like the low-rank factorization of the \( v \) tensor
          yields a non-convex set.
          Can we handle this some way?

    \item Can we relax the constraint on \( \wtwo \) to be the unit ball
          instead of the unit shell?

    \item Is there a better way to express the regularization?
\end{itemize}

\clearpage
\newpage

\printbibliography[]

\end{document}
